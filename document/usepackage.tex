% 枠
\usepackage{fancybox}
\usepackage{ascmac}
% 色
\usepackage{color}
% 数式
\usepackage{amsmath}
\usepackage{amsfonts}
\usepackage{mathtools}
\usepackage{bm,physics}
\usepackage{siunitx}
\usepackage[thicklines]{cancel}
% 画像
\usepackage[dvipdfmx]{graphicx} % 画像出力をする.
% \usepackage[draft]{graphicx} % 画像出力を枠だけにする.
\usepackage[hang,small,bf]{caption}
\usepackage[subrefformat=parens]{subcaption}
\usepackage{here}
\usepackage{wrapfig} % rin; 変更点: 文字の回り込み用のパッケージ.
% グラフ
\usepackage{tikz}
\usetikzlibrary{
  intersections,
  calc,
  arrows.meta
}
% ソースコード
\usepackage{listings,jlisting}
\lstset{
  language=C++,
	stringstyle={\ttfamily},
	commentstyle={\ttfamily},
	basicstyle={\ttfamily},
	columns=fixed,
  frame={tb},
  breaklines=true,
  columns=[l]{fullflexible},
	backgroundcolor=\color[gray]{.90}, % pdfをコピペしたときに行番号を巻き込まないようにする.
  numbers=left, % 行数を表示したければonにする.
  xrightmargin=0em,
  xleftmargin=3em,
  numberstyle={\scriptsize},
  stepnumber=1,
  numbersep=1em,
	tabsize=2,
  lineskip=-0.5ex
}
% アンカー
\usepackage[dvipdfmx]{hyperref}
\usepackage{pxjahyper}
\hypersetup{
  setpagesize=false,
  bookmarksnumbered=true,
  bookmarksopen=true,
  colorlinks=true,
  linkcolor=blue,
  citecolor=red,
  urlcolor=magenta
}
% 数式相互参照
\usepackage{cleveref}
\usepackage{autonum}
\numberwithin{equation}{subsection}
% 目次に参考文献を入れる. これをonにすると目次の体裁が崩れてしまう.
% \usepackage{tocbibind}
