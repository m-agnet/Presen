\documentclass[a4paper,14pt]{jsarticle}

% \vspaceや\textsfで突貫的に作成しているので, 好きに取捨選択してください. (未来のB4へ)

% リスト
\usepackage{enumitem}
% 色
\usepackage{color}
\usepackage{bm}


\begin{document}

\title{
  \vspace{-34mm} 
  {\Large
  \textsf{令和5年度} \\ 
  \textsf{茨城大学 物性理論研究室} \\ 
  \textsf{卒業研究発表会} \\
  \color{red}{
    \textsf{2024年2月1日(木)13:00 - 15:30} \\ 
    \textsf{理学部E棟第6講義室}
    } \\ 
  }
  \vspace{-20mm}
  }
\author{}
\date{}
\pagestyle{empty} % ページ数の非表示.
\maketitle

\subsection*{\textsf{中川研究室: 13:00 - 14:00}}
\begin{itemize}[label={},leftmargin=*]
    \item \textsf{13:00 - 13:15} \quad {仲村友希(Yuki Nakamura)} \item \quad \textbf{マイクロバブルの生成に起因する熱力学量変化}
    \item \textsf{13:15 - 13:30} \quad {平尾崇道(Takamichi Hirao)} 
    \item \quad \textsf{重力による気体・気液共存系の変化に対する圧力を用いたアプローチ}
    \item \textsf{13:30 - 13:45} \quad {山本凜(Rin Yamamoto)} 
    \item \quad \textsf{壁の濡れ性が誘起する,重力と熱流をかけた流体系のダイナミクスの変化}
    \item \textsf{13:45 - 14:00} \quad {金澤広大(Takehiro Kanazawa)} 
    \item \quad \textsf{古典二原子分子の分子状態ゆらぎを特徴づける自由エネルギーの探索}
\end{itemize}

\vspace{3mm}

\centerline{\textsf{15分間の休憩}}

\subsection*{\textsf{福井研究室: 14:15 - 15:30}}
\begin{itemize}[label={},leftmargin=*]
    \item \textsf{14:15 - 14:30} \quad {河村巧(Takumi Kawamura)} 
    \item \quad \textsf{有限温度中の2次元模型のトポロジカル相}
    \item \textsf{14:30 - 14:45} \quad {女屋祐輔(Yusuke Onaya)}  
    \item \quad \textsf{電気回路による格子模型のハミルトニアンの再現}
    \item \textsf{14:45 - 15:00} \quad {中嶋晃介(Kosuke Nakajima)} 
    \item \quad \textsf{電気回路による1次元スピンレス超伝導シミュレーション}
    \item \textsf{15:00 - 15:15} \quad {濱野史奈(Fumina Hamano)} 
    \item \quad \textsf{巻付き数と第1チャーン数の実空間での計算方法}
    \item \textsf{15:15 - 15:30} \quad {伊藤悠哉(Yuya Ito)}  
    \item \quad \textsf{トーリックコードの性質と量子誤り訂正への有用性}
\end{itemize}



\end{document}
